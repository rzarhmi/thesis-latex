
% -------------------------------------------------------
%  Abstract
% -------------------------------------------------------


\pagestyle{empty}

\شروع{وسط‌چین}
\مهم{چکیده}
\پایان{وسط‌چین}
\بدون‌تورفتگی
%نگارش پایان‌نامه‌ علاوه بر بخش پژوهش و آماده‌سازی محتوا،
%مستلزم رعایت نکات فنی و نگارشی دقیقی است 
%که در تهیه‌ی یک پایان‌نامه‌ی موفق بسیار کلیدی و مؤثر است.
%از آن جایی که بسیاری از نکات فنی مانند قالب کلی صفحات، شکل و اندازه‌ی قلم، 
%صفحات عنوان و غیره در تهیه‌ی پایان‌نامه‌ها یکسان است،
%با استفاده از نرم‌افزار حروف‌چینی زی‌تک %\پاورقی{\XeTeX} 
%و افزونه‌ی زی‌پرشین %\پاورقی{XePersian} 
%یک قالب استاندارد برای تهیه‌ی پایان‌نامه‌ها ارائه گردیده است.
%این قالب می‌تواند برای تهیه‌ی پایان‌نامه‌های
%کارشناسی و کارشناسی ارشد و نیز رساله‌ی دکتری مورد استفاده قرار گیرد.
%این نوشتار به طور مختصر نحوه‌ی استفاده از این قالب را نشان می‌دهد.

در سال‌های اخیر، بازشناسی کنش انسان\LTRfootnote{\لر{Human Action Recognition}} از عمده‌ترین زمینه‌های مورد بحث و تحقیق در دنیای علوم و مهندسی کامپیوتر بوده است.\cite{survey} با افزایش چشم‌گیر اطلاعات در دسترس و پیشرفت روزافزون شبکه‌های عصبی\LTRfootnote{Neural Networks}، جلوه‌ی جدیدی به موضوع بازشناسی کنش انسان داده شده است. از مقدمات این موضوع، بحث نحوه‌ی نمایش داده‌های بدن انسان در دو بعد زمان و مکان است. با ظهور سنسورهای کینکت، دو روش عمده برای بهینه‌کردن هرچه بیش‌تر نمایش این داده‌ها وجود دارد. این دو روش شامل استفاده از اطلاعات \لر{RGB-D} و استفاده از اطلاعات سه‌بعدی اسکلت‌های بدن هستند.\cite{review} اخیرا، به‌دلیل کم‌بودن حجم داده‌های اسکلتی، اطلاعات معنائی\LTRfootnote{Semantic Information} بالا و نیز خاصیت مقیاس‌پذیری آن‌ها، مطالعات بسیاری بر روی نمایش داده‌ها بر این روش صورت گرفته است.

در این پروژه مطالعه بر روی نمایش داده‌های سه‌بعدی اسکلتی و استفاده از آن برای بازشناسی کنش انسان ادامه پیدا می‌کند. شبکه‌ی مورداستفاده در این پروژه شبکه‌ی پیچشی‌-گرافی زمان‌-مکانی \LTRfootnote{Spatial Temporal Graph Convolutional Networks} است که تعمیمی بر شبکه‌ی پیچشی‌-گرافی\LTRfootnote{Graph Convolutional Networks} است که آن نیز تعمیمی بر شبکه‌ی پیچشی\LTRfootnote{Convolutional Networks} است. هم‌چنین سعی بر آن بوده است که با ارائه‌ی یک مدل توجه\LTRfootnote{Attention Model} و با استفاده از معیار فلان شبکه‌های عصبی موجود را بهبود بخشید. در کارهای آینده نیز، می‌توان مدل‌های توجه بهینه‌تری را معرفی نمود و تاثیر هرکدام را بر شبکه‌ی مورد استفاده در این پروژه مطالعه نمود.


\پرش‌بلند
\بدون‌تورفتگی \مهم{کلیدواژه‌ها}: 
بازشناسی کنش انسان، اطلاعات سه‌بعدی اسکلت، شبکه‌های گراف‌-پیچشی،‌ مدل توجه
\صفحه‌جدید
