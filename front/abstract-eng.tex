
% -------------------------------------------------------
%  English Abstract
% -------------------------------------------------------


\pagestyle{empty}

\begin{latin}

\begin{center}
\textbf{Abstract}
\end{center}
\baselineskip=.8\baselineskip

Recently, human action recognition have become one of the most studied topics in the world of computer science and engineering. With the enormous increasing of available datasets and the advent of neural networks, new horizons have been opened to the concept of human action recognition.  Representation of human body data is one of the fundamentals of this concept. There are two major methods that’s been introduced to optimize this representation. These two methods largely include using RGB-D data and using 3D skeleton information. Nowadays, lots of research have been taken place on representation of 3D skeleton information, due to its flexibility and lower data size. 

In this project studies towards representation of 3D skeleton information and its use in human action recognition have been continued. Type of network used in this project is spatial temporal graph convolutional network which is a modification of graph convolutional network which is also a modification of convolutional networks. Also, by introducing a beneficial attention model and using filan criteria, current networks have been improved. Future works could introduce more efficient attention models and study their impact on these networks. 

\bigskip\noindent\textbf{Keywords}:
Human Action Recognition, 3D Skeleton Data, Graph Convolutional Networks, Attention Model 

\end{latin}

\newpage
