\chapter{نتایج تجربی}
\section{مجموعه ‌داده‌ی مورداستفاده}


مجموعه ‌داده‌ی \لر{NTU-RGB+D} بزرگ‌ترین مجموعه ‌داده‌ی شامل اطلاعات ۳بعدی اسکلت بدن است.\cite{st-gcn} تصویر \ref{fig:ntu} برخی از قاب‌های موجود در این مجموعه‌ داده را نمایش ‌می‌دهد. این مجموعه داده شامل 56000 کلیپ و 60 دسته کنش است که از \لر{A1} (آشامیدن آب) تا \لر{A60} (از هم فاصله‌گرفتن) برچسب\LTRfootnote{Label}گذاری شده‌اند. این ویدیوها از 3 زاویه‌ی مختلف و با استفاده از ‌سنجش‌گرهای کینکت \LTRfootnote{Kinect Sensors} ضبط شده‌اند تا مختصات ۳بعدی مفاصل به دست آیند. این مجموعه داده به‌صورت کلی به دو دسته سنجه\LTRfootnote{Benchmark} تقسیم شده است.\cite{NTU_paper}
\begin{enumerate}
\item \لر{X-sub}: در این سنجه، بازیگران برای مجموعه‌ی آموزش \LTRfootnote{Train Set} و مجموعه‌ی آزمون \LTRfootnote{Test Set} متفاوت هستند. 
\item \لر{X-view}: در این سنجه، برای مجموعه‌ی آموزش از زاویه‌های ۱ و ۲ دوربین و برای مجموعه‌ی آزمون از زاویه‌ی ۳ دوربین استفاده شده است. 
\end{enumerate}
در این پروژه نیز از هردوی این سنجه‌ها به‌صورت جداگانه استفاده شده است.
\begin{figure}[ht]
\centerimg{ntu_actionRec}{8cm}
\caption[برخی از قاب‌های مجموعه داده‌ی \لر{NTU RGB+D}]{برخی از قاب‌های مجموعه داده‌ی \لر{NTU RGB+D}\cite{ntu_site}}
\label{fig:ntu}
\end{figure}


\قسمت{معیار تابع هزینه‌ی آنتروپی متقابل}
در اکثر مسائل دسته‌بندی، از این تابع به‌عنوان معیار درستی تخمین استفاده می‌کنند.\cite{entropy_src} \cite{st-gcn} در مسائل دسته‌بندی دودویی این تابع با رابطه‌ی 
\begin{equation}\label{eqn:cefbinary}
	L(y, \hat{y}) = - (y\log (\hat{y}) + (1 - y)\log (1 - \hat{y}))
\end{equation}
 تعریف می‌شود.\cite{entropy_src}

برای مسائل دسته‌بندی که تعداد دسته در آن‌ها از دو بیشتر است، این تابع از رابطه‌ی
\begin{equation} \label{eqn:multiclass}
	 L(y, \hat{y}) = - \sum_{i}y_i\log \hat{y_i}
\end{equation}
 محاسبه می‌شود.\cite{entropy_src}

در هر دوی روابط \ref{eqn:cefbinary} و \ref{eqn:multiclass}، $y$ برچسب واقعی است که دو مقدار صفر یا یک را می‌گیرد و $\hat{y}$ مقدار مشاهده‌شده است که هر مقداری بین صفر تا یک می‌تواند بگیرد. 